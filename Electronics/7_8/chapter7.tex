\documentclass[UTF8]{ctexart}
\usepackage{graphicx}
\usepackage{ctex}
\usepackage{tikz}
\usepackage{amsmath}
\title{电动力学-第七次作业}
\author{陈朝翔-2021302021121}


\begin{document}
    \maketitle
    Problem 7.8\\
    Answer:\\
    (a)\\
    The field of long wired is:
    $$\mathbf{B}=\frac{\mu_{0} I}{2 \pi s} \hat{\phi} \eqno(1.1)$$
    so:
    $$\Phi=\int \mathbf{B} \cdot d \mathbf{a}=\frac{\mu_{0} I}{2 \pi} \int_{\mathbf{s}}^{s+a} \frac{1}{s}(a d s)=\frac{\mu_{0} I a}{2 \pi} \ln \left(\frac{s+a}{s}\right) \eqno(1.2)$$
    (b)\\
    $$\mathcal{E}=-\frac{d \Phi}{d t}=-\frac{\mu_{0} I a}{2 \pi} \frac{d}{d t} \ln \left(\frac{s+a}{s}\right) \eqno(1.3)$$
    Since:
    $$\frac{d s}{d t}=v \eqno(1.4)$$
    So:
    $$-\frac{\mu_{0} I a}{2 \pi}\left(\frac{1}{s+a} \frac{d s}{d t}-\frac{1}{s} \frac{d s}{d t}\right)=\frac{\mu_{0} I a^{2} v}{2 \pi s(s+a)} \eqno(1.5)$$
    The field points out of the page, so the force on a charge in the nearby side of the square is to the right. In the far side it's also to the right, but here the field is weaker, so the current flows counterclockwise.\\
    (c)\\
    This time, the flux is constant, so:
    $$\epsilon = 0 \eqno(1.6)$$

    Problem 7.22\\
    Answer:\\
    (a)\\
    From Eq. $5.38,$ the field (on the axis) is:
    $$\mathbf{B}=\frac{\mu_{0} I}{2} \frac{b^{2}}{\left(b^{2}+z^{2}\right)^{3 / 2}} \hat{\mathbf{z}} \eqno(2.1)$$
    So the flux is:
    $$\Phi=\frac{\mu_{0} \pi I a^{2} b^{2}}{2\left(b^{2}+z^{2}\right)^{3 / 2}} \eqno(2.2)$$
    (b)\\
    The field is:
    $$\mathbf{B}=\frac{\mu_{0}}{4 \pi} \frac{m}{r^{3}}(2 \cos \theta \hat{\mathbf{r}}+\sin \theta \hat{\boldsymbol{\theta}}) \eqno(2.3)$$
    And:
    $$m = I \pi a^2 \eqno(2.4)$$
    Integrating over the spherical cap:
    $$\Phi=\int \mathbf{B} \cdot d \mathbf{a}=\frac{\mu_{0}}{4 \pi} \frac{I \pi a^{2}}{r^{3}} \int(2 \cos \theta)\left(r^{2} \sin \theta d \theta d \phi\right)=\frac{\mu_{0} I a^{2}}{2 r} 2 \pi \int_{0}^{\bar{\theta}} \cos \theta \sin \theta d \theta \eqno(2.5)$$
    We have:
    \begin{equation*}
        \left\{
        \begin{aligned}
            &r=\sqrt{b^{2}+z^{2}}\\
            &\sin \bar{\theta}= \frac{b}{r}
        \end{aligned}
        \right.\eqno(2.6)
    \end{equation*}
    So:
    $$\Phi = \frac{\mu_{0} \pi I a^{2} b^{2}}{2\left(b^{2}+z^{2}\right)^{3 / 2}} \eqno(2.7)$$
    This result is same as (a).\\
    (c)\\
    \begin{equation*}
        \left\{
        \begin{aligned}
            &\Phi_{1}=M_{12} I_{2}\\
            &\Phi_{2}=M_{21} I_{1}
        \end{aligned}
        \right.\eqno(2.8)
    \end{equation*}
    So:
    $$M_{12}=M_{21}=\frac{\mu_{0} \pi a^{2} b^{2}}{2\left(b^{2}+z^{2}\right)^{3 / 2}} \eqno(2.9)$$

    Problem 7.40\\
    Answer:\\
    We have:
    $$E=\frac{V}{d} \eqno(3.1)$$
    Then:
    \begin{equation*}
        \left\{
        \begin{aligned}
            &J_c = \sigma E = \frac{V}{\rho d}\\
            &J_d = \frac{\partial D}{\partial t} = \frac{\epsilon V_{0}}{d}[-2 \pi \nu \sin (2 \pi \nu t)]
        \end{aligned}
        \right.\eqno(3.2)
    \end{equation*}
    So:
    $$\frac{J_{c}}{J_{d}}=\frac{V_{0}}{\rho d} \frac{d}{2 \pi \nu \epsilon V_{0}}=\frac{1}{2 \pi \nu \epsilon \rho} \approx 2.41 \eqno(3.3)$$

    Problem 7.44\\
    Answer:\\
    (a)\\
    From Faraday's law:
    $$\nabla \times \mathbf{E}=-\frac{\partial \mathbf{B}}{\partial t} \eqno(4.1)$$
    So:
    $$\frac{\partial B}{\partial t}=0 \eqno(4.2)$$
    Which show that $\mathbf{B(r)}$ is independent of t.\\
    (b)\\
    We have:
    $$\oint \mathbf{E} \cdot d \mathbf{l}=-\frac{d \Phi}{dt} \eqno(4.3)$$
    Here $\mathbf{E} = 0$, so $\Phi$ is constant.\\
    (c)\\
    $$\nabla \times \mathbf{B}=\mu_{0} \mathbf{J}+\mu_{0} \epsilon_{0} \frac{\partial \mathbf{E}}{\partial t} \eqno(4.4)$$
    Since $\mathbf{E} = 0 \text{ and } \mathbf{B} = 0$:
    $$\mathbf{J} = 0 \eqno(4.5)$$
    So, any current must be at the surface.\\
    (d)\\
    From Eq.5.68:
    $$\mathbf{B}=\frac{2}{3} \mu_{0} \sigma \omega a \hat{\mathbf{z}} \eqno(4.6)$$
    So to cancel such a field:
    $$\sigma \omega a=-\frac{3}{2} \frac{B_{0}}{\mu_{0}} \eqno(4.7)$$
    And since:
    $$\mathbf{K}=\sigma \mathbf{v}=\sigma \omega a \sin \theta \hat{\boldsymbol{\phi}} \eqno(4.8)$$
    So:
    $$\mathbf{K}=-\frac{3 B_{0}}{2 \mu_{0}} \sin \theta \hat{\boldsymbol{\phi}} \eqno(4.9)$$

    Problem 7.50\\
    Answer:\\
    Form Eq. 5.3:
    $$qBr = mv \eqno(5.1)$$
    If R stay fixed:
    $$q R \frac{d B}{d t} = qE \eqno(5.2)$$
    We have:
    $$\oint \mathbf{E} \cdot d \mathbf{l}=-\frac{d \Phi}{d t} \eqno(5.3)$$
    Compare (5.2) with (5.3):
    $$-\frac{1}{2 \pi R} \frac{d \Phi}{d t}=R \frac{d B}{d t} \eqno(5.4)$$
    Then:
    $$B=-\frac{1}{2}\left(\frac{1}{\pi R^{2}} \Phi\right) + C \eqno(5.5)$$
    Here C is a constant.\\
    At $t = 0$, the field is off, so $C = 0$, Then:
    $$|B(R)|=\frac{1}{2}\left(\frac{1}{\pi R^{2}} \Phi\right) \eqno(5.6)$$
    So the field at R must be half the average field over the cross-section of the orbit.\\
    Q.E.D

    Problem 8.4\\
    Answer:\\
    (a)\\
    $$(\stackrel{\leftrightarrow}{\mathbf{T}} \cdot d \mathbf{a})_{z}=T_{z x} d a_{x}+T_{z y} d a_{y}+T_{z z} d a_{z} \eqno(1.1)$$
    For x-y plane:
    $$(\stackrel{\leftrightarrow}{\mathbf{T}} \cdot d \mathbf{a})_{z}=\epsilon_{0}\left(E_{z} E_{z}-\frac{1}{2} E^{2}\right)(-r d r d \phi) \eqno(1.2)$$
    Now:
    \begin{equation*}
        \left\{
            \begin{aligned}
                &\mathbf{E}=\frac{1}{4 \pi \epsilon_{n}} 2 \frac{q}{r^{2}} \cos \theta \hat{\mathbf{r}}\\
                &\cos \theta=\frac{r}{s}
            \end{aligned}
        \right.\eqno(1.3)
    \end{equation*}
    So:
    $$E_z = 0 \eqno(1.4)$$
    $$E^2 = \left(\frac{q}{2 \pi \epsilon_{0}}\right)^{2} \frac{r^{2}}{\left(r^{2}+a^{2}\right)^{3}} \eqno(1.5)$$
    Then:
    $$F_{z}=\frac{1}{2} \epsilon_{0}\left(\frac{q}{2 \pi \epsilon_{0}}\right)^{2} 2 \pi \int_{0}^{\epsilon f t y} \frac{r^{3} d r}{\left(r^{2}+a^{2}\right)^{3}} = \frac{q^{2}}{4 \pi \epsilon_{0}} \frac{1}{(2 a)^{2}} \eqno(1.6)$$
    (b)\\
    In this case:
    \begin{equation*}
        \left\{
            \begin{aligned}
                &\mathbf{E}=-\frac{1}{4 \pi \epsilon_{0}} 2 \frac{q}{s^{2}} \sin \theta \hat{\mathbf{z}}\\
                &\sin \theta=\frac{a}{s}
            \end{aligned}
        \right.\eqno(1.7)
    \end{equation*}
    So:
    $$E^{2}=E_{z}^{2}=\left(\frac{q a}{2 \pi \epsilon_{0}}\right)^{2} \frac{1}{\left(r^{2}+a^{2}\right)^{3}} \eqno(1.8)$$
    So:
    $$F_{z}=-\frac{\epsilon_{0}}{2}\left(\frac{q a}{2 \pi \epsilon_{0}}\right)^{2} 2 \pi \int_{0}^{\infty} \frac{r d r}{\left(r^{2}+a^{2}\right)^{3}} = -\frac{q^{2}}{4 \pi \epsilon_{0}} \frac{1}{(2 a)^{2}}\eqno(1.9)$$

    Problem 8.7\\
    Answer:\\
    (a)\\
    $E_{x}=E_{y}=0, E_{z}=-\sigma / \epsilon_{0}$,Then:
    \begin{equation*}
        \left\{
            \begin{aligned}
                &T_{x y}=T_{x z}=T_{y z}=\cdots=0\\
                &T_{x x}=T_{y y}=-\frac{\epsilon_{0}}{2} E^{2}=-\frac{\sigma^{2}}{2 \epsilon_{0}}\\
                T_{z z}=\epsilon_{0}\left(E_{z}^{2}-\frac{1}{2} E^{2}\right)=\frac{\epsilon_{0}}{2} E^{2}=\frac{\sigma^{2}}{2 \epsilon_{0}}
            \end{aligned}
        \right. \eqno(2.1)
    \end{equation*}
    So:
    $$\stackrel{\leftrightarrow}{\mathbf{T}}=\frac{\sigma^{2}}{2 \epsilon_{0}}\left(\begin{array}{ccc}
        -1 & 0 & 0 \\
        0 & -1 & 0 \\
        0 & 0 & +1
        \end{array}\right) \eqno(2.2)$$
    (b)\\
    $$\mathbf{F}=\oint \stackrel{\leftrightarrow}{\mathbf{T}} \cdot d \mathbf{a} \eqno(2.3)$$
    Do integrata over the xy plane;
    $$F_{z}=\int T_{z z} d a_{z}=-\frac{\sigma^{2}}{2 \epsilon_{0}} A \eqno(2.4)$$
    So, the force per unit area is:
    $$f=\frac{\mathbf{F}}{A}=-\frac{\sigma^{2}}{2 \epsilon_{0}} \hat{\mathbf{z}} \eqno(2.5)$$
    (c)\\
    $$-T_{z z}=\frac{\sigma^{2}}{2 \epsilon_{0}} \eqno(2.6)$$
    (d)\\
    The recoil force is the momentum delivered per unit time, so the force per unit area on the top plate is:
    $$f=-\frac{\sigma^{2}}{2 \epsilon_{0}} \hat{\mathbf{z}} \eqno(2.7)$$
    This result is same as (b)\\

    Problem 8.9\\
    Answer:\\
    (a)\\
    The angular momentum stored in the fields is:
    $$\vec{l}=\vec{r} \times \epsilon_{0} \vec{E} \times \vec{B}=\frac{Q B_{0}}{4 \pi r} \hat{r} \times(\hat{r} \times \hat{z}) \eqno(3.1)$$
    We have:
    $$\hat{r} \times(\hat{r} \times \hat{z})=\hat{r}(\hat{r} \cdot \hat{z})-\hat{z}(\hat{r} \cdot \hat{r})=\hat{r} \cos \theta-\hat{z} \eqno(3.2)$$
    Over the sphere, only the z-component of the angular momentum will survive the integration so:
    $$(\hat{r} \times(\hat{r} \times \hat{z}))_{z}=(\hat{r} \cos \theta-\hat{z}) \cdot \hat{z}=\cos ^{2} \theta-1=-\sin ^{2} \theta \eqno(3.3)$$
    Then:
    $$L_{z}=\int_{V} l_{z} d V=\int_{a}^{b} \int_{0}^{\pi} \int_{0}^{2 \pi} \frac{Q B_{0}}{4 \pi r}\left(-\sin ^{2} \theta\right) r^{2} d r \sin \theta d \theta d \phi =-\frac{Q B_{0}}{4 \pi} 2 \pi \int_{a}^{b} r d r \int_{0}^{\pi} \sin ^{3} \theta d \theta\eqno(3.4)$$
    We have:
    $$x=\cos \theta \quad d x=-\sin \theta d \theta \eqno(3.5)$$
    Then:
    $$L_{z}=-\frac{\pi Q B_{0}}{2} \frac{1}{2}\left(b^{2}-a^{2}\right) \int_{1}^{-1} \sin ^{3} \theta \frac{d x}{-\sin \theta} =-\frac{\pi Q B_{0}}{3}\left(b^{2}-a^{2}\right)\eqno(3.6)$$
    So:
    $$\vec{L}=-\frac{\pi Q B_{0}}{3}\left(b^{2}-a^{2}\right) \hat{z} \eqno(3.7)$$
    (b)\\
    When the magnetic field is turning off an electric field is induced. By Faraday's law:
    $$2 E \pi s=-s^{2} \pi \dot{B} \Longrightarrow \vec{E}=-\frac{s}{2} \dot{B} \hat{\phi} \eqno(3.8)$$
    Torque on the patch of surface on a sphere is with (on the sphere) $s=a \sin \theta$:
    $$d \vec{N}=\vec{a} \times d \vec{F}=\vec{a} \times \sigma d S \vec{E}=-\vec{a} \times \sigma d S \frac{a \sin \theta}{2} \dot{B} \hat{\phi} = =-\sigma a^{2} \sin ^{2} \theta d \theta d \phi \vec{a} \times \frac{a}{2} \dot{B} \hat{\phi} \eqno(3.9)$$
    Now, again only the z-component will survive, so the projection of vector $\vec{a} \times \hat{\phi}$ on the z-axis is:
    $$(\vec{a} \times \hat{\phi})_{z}=(\vec{a} \times \hat{\phi}) \cdot \hat{z}=a \cos \left(\pi-\frac{\pi}{2}-\theta\right)=a \sin \theta \eqno(3.10)$$
    With this:
    $$d N_{z}=-\sigma a^{4} \frac{1}{2} \dot{B} \sin ^{3} \theta d \theta d \phi \eqno(3.11)$$
    $$N_{z}=-\frac{1}{2} \dot{B} \sigma a^{4} \int_{0}^{\pi} \sin ^{3} \theta d \theta \int_{0}^{2 \pi} d \phi=-\pi \dot{B} \sigma a^{4} \frac{4}{3} \eqno(3.12)$$
    where we solved the same integral $\left(\sin ^{3} \theta\right)$ in the (a) part of the problem. Now this is valid for both spheres, so for the bigger sphere just replace $a$ with $b .$ Using $\sigma=Q / 4 \pi a^{2}(\text { small sphere })$ and $\sigma=-Q / 4 \pi b^{2}$ the torques are:
    $$N_{z, a}=-\frac{\pi}{3} \dot{B} Q a^{2} \quad N_{z, b}=\frac{\pi}{3} \dot{B} Q b^{2} \eqno(3.13)$$
    The total angular momentum of the system is then:
    $$\begin{aligned}
        L_{z} &=\int_{0}^{t_{f}}\left(N_{z, a}+N_{z, b}\right) d t=\frac{\pi}{3} Q\left(b^{2}-a^{2}\right) \int_{0}^{t_{f}} \dot{B} d t=\frac{\pi}{3} Q\left(b^{2}-a^{2}\right) \int_{B_{0}}^{0} d B \\
        &=-\frac{\pi}{3} Q B_{0}\left(b^{2}-a^{2}\right)
    \end{aligned} \eqno(3.14)$$
    So:
    $$\vec{L}=-\frac{\pi}{3} Q B_{0}\left(b^{2}-a^{2}\right) \hat{z} \eqno(3.15)$$

    Problem 8.14\\
    Answer:\\
    (a)\\
    $$\begin{aligned}
        \frac{U}{l} &=\int_{0}^{2 \pi} \int_{a}^{b} u s d s d \phi \\
        &=\int_{0}^{2 \pi} \int_{a}^{b}\left[\frac{\epsilon_{0}}{2}\left(\frac{\lambda}{2 \pi \epsilon_{0} s}\right)^{2}+\frac{1}{2 \mu_{0}}\left(\frac{\mu_{0} \lambda v}{2 \pi s}\right)^{2}\right] s d s d \phi \\
        &=2 \pi \frac{1}{2 \epsilon_{0}}\left(\frac{\lambda}{2 \pi}\right)^{2} \int_{a}^{b}\left(\frac{1}{s^{2}}+\frac{1}{s^{2}} \frac{v^{2}}{c^{2}}\right) s d s \\
        &=\frac{\lambda^{2}}{4 \pi \epsilon_{0}}\left(1+\beta^{2}\right) \int_{a}^{b} \frac{d s}{s}=\frac{\lambda^{2}}{4 \pi \epsilon_{0}}\left(1+\beta^{2}\right) \ln \frac{b}{a}
        \end{aligned} \eqno(4.1)$$
    (b)\\
    Now, for the momentum per unit lenght stored in the fields just integrate the momentum density over a unit lenght of the system:
    $$\begin{aligned}
        \frac{\vec{p}}{l} &=\int_{0}^{2 \pi} \int_{a}^{b} \vec{g} s d s d \phi=\int_{0}^{2 \pi} \int_{a}^{b} \epsilon_{0} \vec{E} \times \vec{B} s d s d \phi \\
        &=\int_{0}^{2 \pi} \int_{a}^{b} \epsilon_{0} \frac{\lambda}{2 \pi \epsilon_{0} s} \hat{s} \times \frac{\mu_{0} \lambda v}{2 \pi s} \hat{\phi} s d s d \phi \\
        &=\frac{\mu_{0} \lambda^{2} v}{4 \pi^{2}} \int_{a}^{b} \frac{d s}{s} \int_{0}^{2 \pi} d \phi \hat{s} \times \hat{\phi}=\frac{\mu_{0} \lambda^{2} v}{2 \pi} \ln \frac{b}{a} \hat{z}
        \end{aligned}\eqno(4.2)$$
    (c)\\
    For the power transported down the wires integrate the Poynting vector over the annular surface in between the cyllinders, with the positive orientation in the z-direction:
    $$\begin{aligned}
        P=\int_{a}^{b} \int_{0}^{2 \pi} \vec{S} \cdot d \vec{A} &=\frac{1}{\mu_{0}} \int_{a}^{b} \int_{0}^{2 \pi}(\vec{E} \times \vec{B}) \cdot s d s d \phi \hat{z} \\
        &=\frac{1}{\mu_{0}} \int_{a}^{b} \int_{0}^{2 \pi}\left(\frac{\lambda}{2 \pi \epsilon_{0} s} \hat{s} \times \frac{\mu_{0} \lambda v}{2 \pi s} \hat{\phi}\right) \cdot s d s d \phi \\
        &=\frac{\lambda^{2} v}{4 \pi^{2} \epsilon_{0}} \int_{a}^{b} \frac{d s}{s} \int_{0}^{2 \pi} d \phi \\
        &=\frac{\lambda^{2} v}{2 \pi \epsilon_{0}} \ln \frac{b}{a}
        \end{aligned}\eqno(4.3)$$
\end{document}