\documentclass[UTF8]{ctexart}
\usepackage{graphicx}
\usepackage{ctex}
\usepackage{tikz}
\usepackage{amsmath}
\title{电动力学-第十二章作业}
\author{陈朝翔-202130202112}

\begin{document}
	\maketitle
	Problem 12.16\\
	Answer:\\
	(a)\\
	Let the time in the frame of the stationary twin on Earth be $\bar{t}$, then:\\
	$$\begin{aligned}
		\Delta \bar{t} &=\gamma\left(t_{R}+\frac{x_{R} v}{c^{2}}\right)-\gamma\left(t_{S}+\frac{x_{S} v}{c^{2}}\right) \\
		&=\gamma \Delta t=\frac{5}{3} \cdot 18=30 \mathrm{y}
	\end{aligned}$$
	where $x_{R}=x_{S}=0$ since the two origins coincide at start and return. Subscript "R" stands for "return" and "S" for "start". Her twin is thus 51 years old.\\
	(b)\\
	In which system? In the Earth system the star is 15 years of travel away, so it's:
	$$L=15 \cdot \frac{4}{5} c=[12 \mathrm{y}$$
	(c)\\
	since the jump happens at the star $\mathrm{X}$, the coordinates of the jump is the Earth system are obviously $(t, x)=(15 \mathrm{y}, 12 \mathrm{cy})$\\
	Let the Earth system be $\mathcal{S},$ inbound one $\tilde{\mathcal{S}}$ and outbound one $\overline{\mathcal{S}},$ with $t=\bar{t}=\tilde{t}=0$ at $x=\bar{x}=\tilde{x}=0$\\
	(d)\\
	This system travels with her (on the outbound trip), so the jump happens at $(\bar{t}, \bar{x})=(9 \mathrm{y}, 0)$\\
	(e)\\
	Let's use the Lorentz transformations, knowing the coordinates of the jump in the Earth system (from c)). $\tilde{\mathcal{S}}$ moves with respect to Earth system with $-v,$ so:\\
	$$\tilde{x}=\gamma(x+v t)=40 \mathrm{cy} \quad \tilde{t}=\gamma\left(t+x v / c^{2}\right)=41 \mathrm{y}$$
	(f)\\
	She will need to set her clock 32 years foward to agree with the time in $\tilde{\mathcal{S}} .$ since she'll travel for another 9 years (according to her), her clock will read 50 years.\\
	(g)\\
	To the traveling twin the Earth is moving away, so:\\
	$$\bar{t}=\gamma\left(t-x v / c^{2}\right) \Longrightarrow t=\frac{\bar{t}}{\gamma}=5.4 \mathrm{y}$$
	since her brother is at $x=0$ in his origin. Thus, she will observe him being 26.4 years old.\\
	Using the same technique:\\
	$$t=\frac{\tilde{t}}{\gamma}=24.6 \mathrm{y}$$
	so she'll observe him being 45.6 years old.\\
	(h)\\
	Now the Earth is moving towards her at velocity $v,$ so she calulates that on Earth it will pass 5.4 years, just what she calculated for the outbound $\operatorname{trip} \operatorname{in} g, i)$\\
	Adding that to result in $g,$ ii ) she expects that her brother will be 51 years old, just like in a).\\
	
	Problem 12.47\\
	Answer:\\
	(a)\\
	If the system $\bar{S}$ is moving along the $x$ -axis with velocity $v$ we have:\\
	$$\begin{aligned}
	\vec{E}^{\prime} \cdot \vec{B}^{\prime} &=\bar{E}_{x} \bar{B}_{x}+\bar{E}_{y} \bar{B}_{y}+\bar{E}_{z} \bar{B}_{z} \\
	&=E_{x} B_{x}+\gamma\left(E_{y}-v B_{z}\right) \gamma\left(B_{y}+\frac{v E_{z}}{c^{2}}\right)+\gamma\left(E_{z}+v B_{y}\right) \gamma\left(B_{z}-\frac{v E_{y}}{c^{2}}\right)
	\end{aligned}$$
	The mixed products $\left(E_{y} E_{z} \text { and such }\right)$ will cancel, leaving us with the following:\\
	$$\begin{aligned}
		\vec{E}^{\prime} \cdot \vec{B}^{\prime} &=E_{x} B_{x}+E_{y} B_{y} \gamma^{2}\left(1-\beta^{2}\right)+E_{z} B_{z} \gamma^{2}\left(1-\beta^{2}\right) \\
		&=E_{x} B_{x}+E_{y} B_{y}+E_{y} B_{y}=\vec{E} \cdot \vec{B}
	\end{aligned}$$
	since $\gamma^{2}=1 /\left(1-\beta^{2}\right) .$ So it is an invariant under the Lorentzian transformation.\\
	(b)\\
	$$\begin{aligned}
	\bar{E}^{2}-c^{2} \bar{B}^{2}=& \bar{E}_{x}^{2}+\bar{E}_{y}^{2}+\bar{E}_{y}^{2}-c^{2}\left(\bar{B}_{x}^{2}+\bar{B}_{y}^{2}+\bar{B}_{z}^{2}\right) \\
	=& E_{x}^{2}-c^{2} B_{x}^{2}+\gamma^{2}\left(E_{y}-v B_{z}\right)^{2}+\gamma^{2}\left(E_{z}+v B_{y}\right)^{2} \\
	&-c^{2} \gamma^{2}\left(B_{y}+\frac{v E_{z}}{c^{2}}\right)^{2}-c^{2} \gamma^{2}\left(B_{z}-\frac{v E_{y}}{c^{2}}\right)^{2} \\
	=& E_{x}^{2}-c^{2} B_{x}^{2}+\gamma^{2}\left(E_{y}^{2}-2 v E_{y} B_{z}+v^{2} B_{z}^{2}\right)+\gamma^{2}\left(E_{z}^{2}+2 v E_{z} B_{y}+v^{2} B_{y}^{2}\right) \\
	&-c^{2} \gamma^{2}\left(B_{y}^{2}+2 \frac{v}{c^{2}} B_{y} E_{z}+\frac{v^{2}}{c^{4}} E_{z}^{2}\right)-c^{2} \gamma^{2}\left(B_{z}^{2}-2 \frac{v}{c^{2}} B_{z} E_{y}+\frac{v^{2}}{c^{4}} E_{y}^{2}\right)
	\end{aligned}$$
	Again, the mixed products (the middle ones in each of the brackets) will cancel, leaving us with:\\
	$$\begin{aligned}
	\bar{E}^{2}-c^{2} \bar{B}^{2}=& E_{x}^{2}-c^{2} B_{x}^{2}+E_{y}^{2} \gamma^{2}\left(1-\beta^{2}\right)+E_{z}^{2} \gamma^{2}\left(1-\beta^{2}\right) \\
	&-c^{2} B_{y}^{2} \gamma^{2}\left(1-\beta^{2}\right)-c^{2} B_{z}^{2} \gamma^{2}\left(1-\beta^{2}\right) \\
	=& E_{x}^{2}-c^{2} B_{x}^{2}+E_{y}^{2}+E_{z}^{2}-c^{2} B_{y}^{2}-c^{2} B_{z}^{2}=E^{2}-c^{2} B^{2}
	\end{aligned}$$
	So this is an invariant as well.\\
	(c)\\
	In the original system $\vec{E} \neq 0,$ so the invariant b) is greater than zero, and this must be so in all other inertial systems. Therefore, there cannot be a system where the electric field is zero, and magnetic one is not since then the invariant would be negative.\\
	
	Problem 12.51\\
	Answer:\\
	(i)\\
	The first invariant:\\
	$$\begin{array}{c}
	F_{v \mu} F^{\nu \mu} \\
	=F_{00} F^{00}+F_{01} F^{01}+F_{02} F^{02}+F_{03} F^{03} \\
	+F_{10} F^{10}+F_{11} F^{11}+F_{12} F^{12}+F_{13} F^{13} \\
	+F_{20} F^{20}+F_{21} F^{21}+F_{22} F^{22}+F_{23} F^{23} \\
	+F_{30} F^{30}+F_{31} F^{31}+F_{32} F^{32}+F_{33} F^{33}
	\end{array}$$
	The diagonal terms (ones with same indices) dissapear, since they are zero, and:\\
	$$\begin{aligned}
	F_{\nu \mu} F^{\nu \mu} &=-\frac{E_{x}^{2}}{c^{2}}-\frac{E_{y}^{2}}{c^{2}}-\frac{E_{z}^{2}}{c^{2}}-\frac{E_{x}^{2}}{c^{2}}+B_{z}^{2}+B_{y}^{2}-\frac{E_{y}^{2}}{c^{2}}+B_{z}^{2}+B_{x}^{2}-\frac{E_{z}^{2}}{c^{2}}+B_{y}^{2}+B_{x_{x}}^{2} \\
	&=-\frac{2}{c^{2}}\left(E^{2}-c^{2} B^{2}\right)
	\end{aligned}$$
	This is the same (apart from a constant factor) invariant from the Prob 12.47)\\
	(ii)\\
	The second invariant:\\
	$$\begin{array}{c}
	G_{\nu \mu} G^{\nu \mu} \\
	=G_{00} G^{00}+G_{01} G^{01}+G_{02} G^{02}+G_{03} G^{03} \\
	+G_{10} G^{10}+G_{11} G^{11}+G_{12} G^{12}+G_{13} G^{13} \\
	+G_{20} G^{20}+G_{21} G^{21}+G_{22} G^{22}+G_{23} G^{23} \\
	+G_{30} G^{30}+G_{31} G^{31}+G_{32} G^{32}+G_{33} G^{33}
	\end{array}$$
	Agian, the diagonal terms are zero, and we are left with:\\
	$$\begin{aligned}
		G_{\nu \mu} G^{\nu \mu} &=-B_{x}^{2}-B_{y}^{2}-B_{z}^{2}-B_{x}^{2}+\frac{E_{z}^{2}}{c^{2}}+\frac{E_{y}^{2}}{c^{2}}-B_{y}^{2}+\frac{E_{z}^{2}}{c^{2}}+\frac{E_{x}^{2}}{c^{2}}-B_{z}^{2}+\frac{E_{y}^{2}}{c^{2}}+\frac{E_{x}^{2}}{c^{2}} \\
		&=\frac{2}{c^{2}}\left(E^{2}-c^{2} B^{2}\right)
	\end{aligned}$$
	The same invariant as in the first part of the problem, with a reversed sign, so:\\
	$$G_{\nu \mu} G^{\nu \mu}=-F_{\nu \mu} F^{\nu \mu}$$
	(iii)\\
	The final invariant:\\
	$$\begin{array}{c}
		G_{\nu \mu} F^{\nu \mu} \\
		=G_{00} F^{00}+G_{01} F^{01}+G_{02} F^{02}+G_{03} F^{03} \\
		+G_{10} F^{10}+G_{11} F^{11}+G_{12} F^{12}+G_{13} F^{13} \\
		+G_{20} F^{20}+G_{21} F^{21}+G_{22} F^{22}+G_{23} F^{23} \\
		+G_{30} F^{30}+G_{31} F^{31}+G_{32} F^{32}+G_{33} F^{33}
	\end{array}$$
	Diagonal terms are zero:\\
	$$\begin{aligned}
	G_{\nu \mu} F^{\nu \mu}=&-\frac{E_{x}}{c} B_{x}-\frac{E_{y}}{c} B_{y}-\frac{E_{z}}{c} B_{z}-\frac{E_{x}}{c} B_{x}-\frac{E_{z}}{c} B_{z}-\frac{E_{y}}{c} B_{y} \\
	&-\frac{E_{y}}{c} B_{y}-\frac{E_{z}}{c} B_{z}-\frac{E_{x}}{c} B_{x}-\frac{E_{z}}{c} B_{z}-\frac{E_{y}}{c} B_{y}-\frac{E_{x}}{c} B_{x} \\
	=&-\frac{4}{c}\left(E_{x} B_{x}+E_{y} B_{y}+E_{z} B_{z}\right)=-\frac{4}{c} \vec{E} \cdot \vec{B}
	\end{aligned}$$
	This is also an invariant, just like in Problem 12.47 ), apart from the constant factor.
	
\end{document}